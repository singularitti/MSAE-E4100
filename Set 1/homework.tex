\documentclass[openany]{homework}

\usepackage[sc]{mathpazo} % use mathpazo for math fonts
\usepackage{tgpagella} % but use tgpagella as main font
\usepackage{enumerate}
\usepackage{multirow}
\usepackage{bm}
\usepackage{miller}
\usepackage{siunitx}

\coursename{MSAE E4100 Crystallography} % DON'T CHANGE THIS

\studname{Qi Zhang}           % YOUR NAME GOES HERE
\studmail{qz2280@columbia.edu}       % YOUR UNI GOES HERE
\hwNo{1}                          % THE HOMEWORK NUMBER GOES HERE


\begin{document}
\maketitle

\section*{1}
\begin{enumerate}[a.]
        \item A four-fold rotation about the $\bm{a}$-axis is $4 \hkl[1 0 0]$, apply it thrice will be $4^3 \hkl[1 0 0]$.
        \item A six-fold rotoinversion about the $\bm{c}$-axis is $\bar{6} \hkl[0 0 1]$, apply it thrice will be $\bar{6}^3 \hkl[0 0 1] = m \hkl[0 0 1]$.
        \item A mirror operation containing both the $\bm{b}$- and $\bm{c}$-axes in hexagonal coordinates will be
              $m \hkl[2 1 0]$. As is drawn below.
\end{enumerate}

\newpage

\section*{2}
\begin{enumerate}[a.]
        \item A six-fold rotation about the $\bm{c}$-axis is $C_6 \hkl[0 0 1]$, apply it thrice will be $C_6^2 \hkl[0 0 1] = C_3 \hkl[0 0 1]$.
        \item A four-fold rotation-reflection with rotation about and reflection perpendicular to
              the $\bm{c}$-axis is $S_4 \hkl[0 0 1]$, apply it thrice will be $S_4^2 \hkl[0 0 1] = C_2 \hkl[0 0 1]$.
        \item A mirror plane perpendicular to the $\bm{c}$-axes is $\sigma_h \hkl[0 0 1]$.
        \item A two-fold rotation about the $\bm{a}$-axis is $C_2 \hkl[1 0 0]$.
\end{enumerate}

\section*{3}
\begin{enumerate}[a.]
        \item For a six-fold rotation axis, each operation will rotate the original point by \SI{60}{\degree}, so it requires $4$ times of rotation.
              That is to say, the matrix is
              \begin{equation}
                      M = \begin{bmatrix}
                              1 & \bar{1} & 0 \\
                              1 & 0       & 0 \\
                              0 & 0       & 1
                      \end{bmatrix}^4 =
                      \begin{bmatrix}
                              \bar{1} & 1 & 0 \\
                              \bar{1} & 0 & 0 \\
                              0       & 0 & 1
                      \end{bmatrix}.
              \end{equation}
              Afterthat, by reflected in a plane perpendicular to the $\bm{a}$-axis, the corresponding matrix is
              $R = \left[\begin{smallmatrix}
                                      \bar{1} & 0 & 0\\
                                      0           & 1 & 0\\
                                      0           & 0 & 1
                              \end{smallmatrix}\right]$,
              thus the whole matrix will be
              \begin{equation}
                      M' = M R =\begin{bmatrix}
                              \bar{1} & 1 & 0 \\
                              \bar{1} & 0 & 0 \\
                              0       & 0 & 1
                      \end{bmatrix}
                      \begin{bmatrix}
                              \bar{1} & 0 & 0 \\
                              0       & 1 & 0 \\
                              0       & 0 & 1
                      \end{bmatrix} =
                      \begin{bmatrix}
                              1 & 1 & 0 \\
                              1 & 0 & 0 \\
                              0 & 0 & 1
                      \end{bmatrix}.
              \end{equation}
              So a general point $(x, y, z)$ will be sent to $(x+y, x, z)$.
        \item The mirror plane $m \hkl[-1 1 0]$ and $m \hkl[1 -1 0]$ are equivalent, so they have the same matrix representation
              $M = \left[\begin{smallmatrix}
                                      0 & 1 & 0\\
                                      1 & 0 & 0\\
                                      0 & 0 & 1
                              \end{smallmatrix}
                              \right]$, the matrix of rotation by \SI{180}{\degree} is
              $R = \left[\begin{smallmatrix}
                                      0 & 1 & 0\\
                                      1 & 0 & 0\\
                                      0 & 0 & \bar{1}
                              \end{smallmatrix}
                              \right]$, thus the whole matrix is
              \begin{equation}
                      M' = M R = \begin{bmatrix}
                              1 & 0 & 0 \\
                              0 & 0 & 1 \\
                              0 & 1 & 0
                      \end{bmatrix}
                      \begin{bmatrix}
                              0 & 1 & 0       \\
                              1 & 0 & 0       \\
                              0 & 0 & \bar{1}
                      \end{bmatrix} =
                      \begin{bmatrix}
                              1 & 0 & 0       \\
                              0 & 1 & 0       \\
                              0 & 0 & \bar{1}
                      \end{bmatrix}.
              \end{equation}
              So a general point $(x, y, z)$ will be sent to $(x, y, -z)$.
\end{enumerate}

\section*{4}
\begin{enumerate}[a.]
        \item The matrix is
              \begin{equation}
                      M = M_1 M_2 =
                      \begin{bmatrix}
                              1 & 0 & 0       \\
                              0 & 1 & 0       \\
                              0 & 0 & \bar{1}
                      \end{bmatrix}
                      \begin{bmatrix}
                              0       & \bar{1} & 0 \\
                              \bar{1} & 0       & 0 \\
                              0       & 0       & 1
                      \end{bmatrix} =
                      \begin{bmatrix}
                              0       & \bar{1} & 0       \\
                              \bar{1} & 0       & 0       \\
                              0       & 0       & \bar{1}
                      \end{bmatrix}.
              \end{equation}
              So a general point $(x, y, z)$ will be sent to $(-y, -x, -z)$. The pattern is drawn below.
        \item Yes. The matrix corresponds to $2 \hkl[1 -1 0]$. And the symmetry element is a two-fold rotation axis.
        \item The matrix of the axis is
              $\left[\begin{smallmatrix}
                                      \bar{1} & 0 & 0\\
                                      0           & \bar{1} & 0\\
                                      0              & 0     & 1
                              \end{smallmatrix}\right]$,
              and the matrix of the mirror is
              $\left[\begin{smallmatrix}
                                      1 & 0 & 0\\
                                      0 & 1 & 0\\
                                      0 & 0 & \bar{1}
                              \end{smallmatrix}\right]$. Thus the whole matrix is
              \begin{equation}
                      M = M_1 M_2 =
                      \begin{bmatrix}
                              \bar{1} & 0       & 0 \\
                              0       & \bar{1} & 0 \\
                              0       & 0       & 1
                      \end{bmatrix}
                      \begin{bmatrix}
                              1 & 0 & 0       \\
                              0 & 1 & 0       \\
                              0 & 0 & \bar{1}
                      \end{bmatrix} =
                      \begin{bmatrix}
                              \bar{1} & 0       & 0       \\
                              0       & \bar{1} & 0       \\
                              0       & 0       & \bar{1}
                      \end{bmatrix}.
              \end{equation}
              Thus a general point $(x, y, z)$ will be sent to $(-x, -y, -z)$. The pattern is drawn below.
        \item Yes. The symmetry operator is $i$. The center of symmetry is $(0, 0, 0)$.
\end{enumerate}

\end{document}